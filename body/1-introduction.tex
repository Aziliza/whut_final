\newpage
\setcounter{page}{1}
\renewcommand\thesection{\arabic{section}}
\titleformat{\section}{\centering \zihao{-2}\heiti}{第 \thesection 章}{1em}{}
% \titleformat*{\section}{\zihao{3}\heiti}
\titleformat*{\subsection}{\zihao{3}\heiti}
\titleformat*{\subsubsection}{\zihao{4}\heiti}
\section{绪论}

\subsection{研究背景与意义}
流行病(epidemic)是指某种传染病在一定时期和范围内出现的高发病率现象,通常会导致人群中较高的感染率和死亡率。流行病往往具有暴发性和快速传播的特点,可能对社会和经济造成较大的影响。传染病一直以来都是威胁着人们身体健康的最大敌人,在历史上屡屡发生的疫情,对人们的生活、国家的民生都造成了极大的影响。公元2世纪,罗马的安东尼之灾造成了人口锐减,经济衰败,给外敌以可乘之机,致使罗马帝国灭亡\ucite{2003Mathematical}。公元1519年到1530年间麻疹等传染病的流行,使墨西哥的印第安人从3000万下降到300万。令人闻风丧胆的黑死病瘟疫,曾经在欧洲发生过四次大的瘟疫:第一次发生在公元600年,它夺去了欧洲半数的人口,最严重的一次是一天一万人以上\ucite{皮特·布鲁克史密斯1999};第二次是在1346年到1350年之间,造成欧洲三分之一的人丧生;第三次是在1665年到1666年,造成了伦敦六分之一的人的死亡\ucite{2003Mathematical}。近年来,我国传染病的流行形势正变得越来越严峻,如SARS、H7N9禽流感、新冠肺炎、甲型流感等疫情的爆发,给社会经济发展和人民群众的健康带来了极大的危害。传染病的防控是国家公共卫生安全的重要组成部分,而传染病的传播动力学研究是防控传染病的重要基础。

研究流行病传播的目的是为了更好地了解疾病的传播规律和机制,从而预测疾病的发展趋势、控制疫情的扩散、制定有效的防控措施,以保障公共卫生安全和人民群众的健康。研究流行病传播可以建立数学模型,通过对模型参数的拟合和模拟计算,预测疫情的发展趋势,为防控措施的制定提供科学依据。促进防疫工作的规划和管理。通过对流行病传播机制的研究,可以发现和防控传染病的薄弱环节,加强监测和预警,制定针对性的防控措施,提高防疫工作的效率和质量。推动防疫技术的创新和进步。研究流行病传播可以提高人们对传染病的认知和了解,推动防疫技术的创新和进步,提高防疫技术的可靠性和实用性,促进公共卫生事业的发展,保障人民群众的健康和安全。

\subsection{国内外研究现状}
当前,对于传染性疾病的研究根据分析方式的不同可以大致分为四大类:理论研究、分析性研究、描述性研究、实验性研究。在对传染性疾病进行量化分析的基础上,建立了传染病动力学模型。主要研究内容包括:基于传染病的群体增长特征、疫情的发生、传播、发展规律及其相关的社会关系等,构建能够体现传染病的动态特征的数学模型;对该模型的动态行为进行定性、定量分析,并结合数值仿真,研究该模型的动态行为,揭示该疾病的传播规律,预测该疾病的发展趋势,分析该疾病的成因,寻找该疾病的防控措施,为该疾病的防控提供科学的理论依据。相较于传统的统计方法,流行病动力学方法能更好地根据感染人数、死亡人数、流行区域等数据从疾病的传播机理方面来反映流行规律,从而能使人们发现疾病爆发过程中的一些流行特点。通过将流行病动力学与生物统计学、计算机模拟仿真、数学建模等方法结合,在相关交叉领域进行研究,能使人们更加深入和全面地认识传染病流行规律,从而建立更加符合实际的理论,提出更加可靠和符合实际的防治策略。

\subsubsection{流行病动力学模型}
微分方程模型是传统的流行病动力学模型,主要基于人群的总体变化来描述疾病传播的规律。其中,最为典型的模型是SIR模型和SEIR模型,两种模型都对人群做了划分,并且使用了一组微分方程来描述不同群体之间的转换,这两种模型都有着不错的预测效果。SIR模型最早由英国统计学家Kermack和McKendrick在1927年提出,用于研究传染病在人群中传播的过程。他们的研究旨在探讨如何利用数学方法对传染病进行预测和控制。在研究中,他们将人群分为三个状态,即易感者、感染者和康复者,并用微分方程描述它们之间的转化关系。该模型被称为SIR模型,S表示易感者(Susceptible)、I表示感染者(Infected)、R表示康复者(Recovered)。SEIR模型是对SIR模型的改进,它增加了一个暴露者(Exposed)的状态,用来描述那些已经被感染但尚未表现出症状的人。该模型可以更准确地描述传染病的传播过程。

此外,还有在微分方程的基础之上针对传染病的时滞性、迁移性、年龄差异性做出改进的模型。目前,SIR模型及其衍生模型已成为流行病学研究的常用工具,广泛应用于疾病预测、控制和公共卫生决策。它们不仅可以对传染病进行建模和模拟\ucite{王健2020基于},还可以评估不同防控措施的效果\ucite{王寅2020基于},为制定有效的疾病防控策略提供科学依据\ucite{李承倬2020基于}。但是,该模型也有一定局限性,例如它假设人口总量不变、传染病只通过直接接触传播等,无法考虑更加复杂的疾病传播情况。

D. Bernoulli在1760年就利用数学方法对天花的扩散进行了研究,可以说从20世纪开始,人们就对传染病进行确定性研究了。1906年, Hamer为了研究麻疹的反复流行,构建并分析了一个离散时间模型\ucite{1979Epidemic}。1911年,公共卫生医生Ross博士为了研究疟疾在蚊子和人类之间传播的动力学行为,使用微分方程模型对其在时间与数量上的变化进行了分析\ucite{1911The},最后,他得出的结论是,只要把蚊虫的数目降低到一个阈值之下,就可以有效地控制疟疾的流行。Ross的这项研究使他第二次获得了Nobel医学奖。1926年Kermack与McKendrick为了研究1665年到1666年黑死病在伦敦的流行规律,并且再次基础上进一步分析了1906年孟买瘟疫的流行规律,构造了著名的SIR仓室模型\ucite{1927A}。这一模型的提出为之后流行病动力学模型打下了基础。在此之后,他们又在1932年改进了SIS仓室模型\ucite{W1991Contributions},并使用收集的数据在之前建立模型的基础上,提出了“阈值理论”,这一理论能够区分疾病是否能够在某一地区长期流行。有了他们为流行病动力学研究打下的坚实基础,Bailey于1957年出版了专著\ucite{1977The},在这一专著中他总结了前人模型的经验和特点,从此20世纪中叶流行病动力学的建模与研究开始蓬勃地发展。

比较方法是加深对疾病传播过程理解的一种途径。通过比较同一种疾病在不同时间或不同人群中的情况,或比较不同疾病在同一种群体中的表现,可以深入了解这些疾病的关键参数。例如,May\ucite{1993The}和 Hethcote\ucite{1977Present} 对于麻疹、百日咳、水痘、白喉、腮腺炎、德国麻疹、小儿麻痹症、天花等疾病的基本再生数进行了估计,并通过这些再生数推断出为了获得群体免疫,人群必须接受免疫接种的比例。

此外,控制和优化方法也被广泛用于传染病模型的研究。Wickwire 在1977年的文章中概述了传染病控制方面的建模\ucite{Anon1996World}。Hethcote 和 Waltman 在1973年就已经开始使用动力学方法寻求最优的控制疾病流行的接种策略,以最小化花费\ucite{Herbert1973Optimal}。Longini 等人在1978年确定了有限的接种资源下,对于香港和亚洲流感的最佳年龄和社会群体\ucite{1978An}。在1988年,Hethcote 进行了对麻疹的研究,在三个地理区域找到了最理想的接种最佳年龄 \ucite{1988Optimal}。

\subsubsection{神经网络和个体模型}

近年来,流行病学研究者已开始运用神经网络和微分方程等方法,以更精准地预测和控制传染病的传播。人工神经网络(Artificial Neural Networks,ANN)是解决多节点和多输出实际问题的网络结构。虽然人工神经网络和人类大脑都能强大地处理信息,但它们之间有许多差异。谷歌DeepMind的Demis Hassabis、Mustafa Suleyman和Shane Legg在2016年成功开发了AlphaGo,击败了世界围棋冠军李世石,这表明人工神经网络有巨大的潜力。机器人使用人工神经网络进行线性处理以处理信息,计算机能够在串行算术任务上快速准确地超越人类。然而,相对于人工神经网络,人类大脑的“并行处理体系”具有绝对优势。ANN通常由多层神经元组成,包括输入层、隐藏层和输出层。输入层接收外部输入信号,隐藏层和输出层进行一系列的计算和传递,最终输出预测或分类结果。ANN通过反向传播算法学习权重和偏差参数,以最小化预测或分类误差。例如,神经微分方程模型可以将微分方程模型中的演化算子替换为神经网络,以更好地适应现实中的不确定性和非线性特征\ucite{2023Optimal}。这些新型模型为研究者提供了更多的选择,使得我们能够更好地探究疾病传播的规律。

个体级模型是基于个体之间的直接相互作用来描述传染病传播的规律。这类模型主要包括基于代理人的模型(Agent-based model)和基于网络的模型(Network-based model)两种。

其中,代理人模型(ABM)是一种基于个体行为和交互的建模方法,用于模拟社会、生态、经济、生物等复杂系统的演化过程。在ABM中,系统中的每个个体都被看作是一个自主的决策制定者,具有一定的属性、行为和互动规则,它们可以根据自身的状态和环境信息进行决策和行动,并与其他个体互动。通过模拟这些个体之间的交互和演化,ABM可以模拟系统整体的行为和演化过程,从而预测系统的动态变化和评估不同决策对系统的影响。在流行病学研究中,ABM可以用于研究传染病的传播过程和防控策略的效果。通过建立每个人作为一个个体,考虑其个体属性、行为和交互规则,可以更加真实地模拟传染病在人群中的传播过程。ABM还可以用于评估不同的防控措施的效果,如隔离、口罩、社交距离等措施,以便制定更加有效的疾病防控策略。

网络模型(NBM)是一种基于网络结构的建模方法,用于研究复杂系统的行为和演化过程。在这种模型中,系统中的每个元素(如人、节点、分子等)被看作是网络中的一个节点,节点之间的联系和关系则用网络中的边来表示。通过建立网络结构,并考虑节点之间的相互作用和影响,可以模拟系统的演化和行为,预测系统的动态变化和评估不同决策对系统的影响。在流行病学研究中,NBM可以用于研究传染病在网络中的传播过程和防控策略的效果\ucite{Chun2015Dynamics}。例如,可以将每个人群聚集区看作是一个节点,将不同的社交关系和人群迁移行为看作是网络中的边,建立一个社交网络,模拟传染病在网络中的传播过程。这种模型可以考虑节点之间的不同属性和行为,以及节点之间的关系和联系,更好地模拟传染病在实际社交网络中的传播过程。与传统的流行病模型相比,NBM可以更好地考虑节点之间的相互作用和影响,更真实地反映实际的传染病传播过程。但是,这种模型也存在着一些挑战,如需要大量的数据和计算资源,以及难以验证和复制模型等问题。

近年来,研究者们开始将神经网络、微分方程等方法引入流行病动力学中,以期能够更准确地预测和控制传染病的传播。人工神经网络(Artificial Neural Network)是一种计算机模型,受到生物神经元网络的启发而设计。ANN通常由多层神经元组成,包括输入层、隐藏层和输出层。输入层接收外部输入信号,隐藏层和输出层进行一系列的计算和传递,最终输出预测或分类结果。ANN通过反向传播算法学习权重和偏差参数,以最小化预测或分类误差。例如,神经微分方程模型可以将微分方程模型中的演化算子换成神经网络,使得模型能够更好地适应现实中的不确定性和非线性特征\ucite{2023Optimal}。这些新型模型为研究者提供了更多的选择,使得我们能够更好地探究疾病传播的规律。

\subsection{本文的研究内容}
由于流行病传播的复杂性和多样性,基于传染病动力学的传统模型有着一定的局限性,而基于机器学习的神经网络模型往往需要大量的训练数据才能有着较好的预测效果。针对新形势下疫情传播的复杂性和突发性,本文将基于机器学习的神经网络引入流行病动力学模型。神经网络能够通过多层神经元的相互连接,表现出高度的非线性特性。这种非线性特性使得神经网络能够更好地捕捉复杂系统的行为,包括流行病学中的疾病传播过程。同时保留了SIER模型中人群的转换关系,从而更好地对现实问题进行预测和拟合。具体包括以下几个方面:

SEIR模型的建立和分析:介绍传染病动力学中的基本再生数、疾病发生率、有效接触数、修正接触数、有效接触率、净增长阈值、流行周期等基本概念。介绍传统的SIR模型,针对不同参数对其全局稳定性的影响进行分析,并考虑具有不同出生数和疾病发生率的SEIR模型。

神经网络的研究:介绍神经网络的基本概念,包括神经元、权重、激活函数、前向传播、反向传播等内容,并探讨神经网络,尤其是循环神经网络在时序性数据预测和拟合方面的应用。

神经微分方程:基于微分方程的神经网络模型,将神经网络嵌入微分方程,利用神经网络高度的非线性特性,对微分方程中的某一结构进行替代。分析这一方法的可行性,并对如何在微分方程中使用神经网络的反向传播进行了理论分析。

实验与分析:使用具有恒定参数和可变参数的SEIR数据对神经微分方程进行实验,对实验结果进行分析并与其他算法进行对比。使用美国安大略省的COVID-19真实数据进行实验,并将预测数据与真实数据进行对比。